%% REMOVED %%

%%%%%%%%%%%%%%%%%%%%%%%%%%%%%%%%%%%%%%%%%%%%%%%%%%%%%%%
\subsection{Lifting over Product Spaces}


The algorithms defined through iterations~\eqref{eq-iter-dystra} and~\eqref{eq-iter-bregmanproj} are not symmetric with respect to the indexing of the constraints $\{\Cc_n\}$, i.e. if we re-order the constraints, the trajectories $\{\pi_n\}_n$ differ.

A way to obtain symmetric iterations can be achieved by lifting the problem over the product space $\bpi = (\pi^\ell)_{\ell=1}^L \in (\Si_N)^L$ and re-cast~\eqref{proj-inter} as the following projection
\eql{\label{eq-projinter-lifted}
	\min \enscond{
		\KLdivL{\bpi}{\bar\bpi} = \sum_{\ell=1}^L \la_\ell \KLdiv{\pi^\ell}{\bar\pi^\ell}
	}{
		\bpi \in \bCc_1 \cap \bCc_2
	}
}
where we introduced a a set of weights $\la \in \RR_+^L$, normalized so that $\sum_\ell \la_\ell=1$.

It is easy to check that similarly to the KL divergence, $\KL_\la$ is a also a $f$-divergence as defined in~\cite{}.

The lifted constraint sets are defined as
\begin{align*}
	\bCc_1 &= \enscond{ 
		\bpi = (\pi^\ell)_{\ell=1}^L \in (\Si_N)^L
	}{
		\forall \ell, \ell', \; \pi^\ell = \pi^{\ell'}
	} \\
	\bCc_2 &= \enscond{ 
		\bpi = (\pi^\ell)_{\ell=1}^L \in (\Si_N)^L
	}{
		\forall \ell, \; \pi^\ell \in \Cc_\ell
	} .
\end{align*}

The projection on $\bCc_2$ is simple to compute since the constraint is separable
\eq{
	\KLprojL_{\bCc_2}(\bar\bpi)
	= \pa{ 
		\KLproj_{\Cc_\ell}(\bar\pi_\ell)
	}_{\ell=1}^L
}
The projection on the diagonal set $\bCc_1$ is computed as described in the following proposition. 

\begin{prop}
	The projection $(\pi^\ell)_{\ell=1}^L = \KLprojL_{\Cc_1}(\bar\bpi)$ satisfies 
	\eq{
		\foralls \ell=1,\ldots,L, \quad
			\pi^\ell = \prod_{r=1}^L (\pi^r)^{\la_r}
	}
	where product and exponentiation should be understood component wise. 
\end{prop}
\begin{proof}
	TODO.
\end{proof}

It is then possible to use iterations~\eqref{eq-iter-bregmanproj} (when the sets $\Cc_\ell$ are affine) or \eqref{eq-iter-dystra} (for the general case) to solve the projection~\eqref{eq-projinter-lifted}, which defines iterations that differs from the one obtained when applying these algorithm directly to~\eqref{proj-inter}.